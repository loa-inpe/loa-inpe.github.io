%!TEX TS-program = xelatex
%!TEX encoding = UTF-8 Unicode
% Awesome CV LaTeX Template for Cover Letter
%
% This template has been downloaded from:
% https://github.com/posquit0/Awesome-CV
%
% Authors:
% Claud D. Park <posquit0.bj@gmail.com>
% Lars Richter <mail@ayeks.de>
%
% Template license:
% CC BY-SA 4.0 (https://creativecommons.org/licenses/by-sa/4.0/)
%-------------------------------------------------------------------------------
% CONFIGURATIONS
%-------------------------------------------------------------------------------
% A4 paper size by default, use 'letterpaper' for US letter
\documentclass[11pt, a4paper]{awesome-cv}
\usepackage{hyperref}
\definecolor{bleu_cite}{RGB}{12,127,172}

% Configure page margins with geometry
\geometry{left=1.4cm, top=.8cm, right=1.4cm, bottom=1.8cm, footskip=.5cm}

% Specify the location of the included fonts
\fontdir[fonts/]

% Color for highlights
\colorlet{awesome}{awesome-skyblue}

% Set false if you don't want to highlight section with awesome color
\setbool{acvSectionColorHighlight}{true}

% If you would like to change the social information separator from a pipe (|) to something else
\renewcommand{\acvHeaderSocialSep}{\quad\textbar\quad}

%-------------------------------------------------------------------------------
%	PERSONAL INFORMATION
%	Comment any of the lines below if they are not required
%-------------------------------------------------------------------------------
% Available options: circle|rectangle,edge/noedge,left/right
\photo[rectangle,noedge,left]{logo}
\name{Laboratório de Estudos do Oceano e da Atmosfera}{}
\homepage{www.loainpe.com.br}

%------------------------------------------------------------------------------
%	LETTER INFORMATION
%	All of the below lines must be filled out
%-------------------------------------------------------------------------------
% The company being applied to
\recipient
  {Produção bibliográfica do Laboratório de Estudos do Oceano e da Atmosfera}
  {Última atualização: 30 de Março de 2020}

% The date on the letter, default is the date of compilation
\letterdate{ }
% The title of the letter
\lettertitle{}
% How the letter is opened
\letteropening{}


%-------------------------------------------------------------------------------
\begin{document}

% Print the header with above personal informations
% Give optional argument to change alignment(C: center, L: left, R: right)
\makecvheader[R]

% Print the title with above letter informations
\makelettertitle

%-------------------------------------------------------------------------------
%	LETTER CONTENT
%-------------------------------------------------------------------------------
\begin{cvletter}
\vspace{-2cm}
\lettersection{Artigos revisados ​​por pares}
\begin{itemize}
  \item[] SOUZA, RONALD ; PEZZI, LUCIANO ; SWART, SEBASTIAAN ; OLIVEIRA, FABRÍCIO ; SANTINI, MARCELO .
  Air-Sea Interactions over Eddies in the Brazil-Malvinas Confluence. Remote Sensing, v. 13, p. 1335,
  doi:\textcolor{bleu_cite}{\href{https://doi.org/10.3390/rs13071335}{10.3390/rs13071335}}, 2021.

  \item[] VOERMANS, JOEY J. ; BABANIN, ALEXANDER V. ; KIREZCI, CAGIL ; CARVALHO, JONAS T. ; SANTINI, MARCELO F. ; PAVANI, BRUNA F. ; Pezzi, Luciano P.  
  Wave Anomaly Detection in Wave Measurements. JOURNAL OF ATMOSPHERIC AND OCEANIC TECHNOLOGY, v. 38, p. 525-536, 
  doi:\textcolor{bleu_cite}{\href{https://doi.org/10.1175/JTECH-D-20-0090.1}{10.1175/JTECH-D-20-0090.1}}, 2021.
 
  \item[] COMIN, ALCIMONI NELCI ; Justino, Flavio ; PEZZI, LUCIANO ; DE SOUSA GURJÃO, CARLOS DIEGO ; SHUMACHER, VANÚCIA ; FERNÁNDEZ, ALFONSO ; SUTIL, UESLEI ADRIANO . 
  Extreme rainfall event in the Northeast coast of Brazil: a numerical sensitivity study. Meteorology and Atmospheric Physics, v. 1, p. 1-22,  
  doi:\textcolor{bleu_cite}{\href{https://doi.org/10.1007/s00703-020-00747-0}{10.1007/s00703-020-00747-0}}, 2020.

  \item[] SANTINI, M. F. ; SOUZA, R. B. ; Pezzi, L. P. ; SWART, S. . Observations of air-sea heat 
  fluxes in the southwestern Atlantic under high-frequency ocean and atmospheric perturbations. Quarterly Journal of the Royal Meteorological Society, v. 1, p. qj.3905, 
  doi:\textcolor{bleu_cite}{\href{https://doi.org/10.1002/qj.3905}{10.1002/qj.3905}}, 2020.

  \item[] FRANCO, BÁRBARA C. ; DEFEO, OMAR ; PIOLA, ALBERTO R. ; BARREIRO, MARCELO ; YANG, HU ; ORTEGA, LEONARDO ; GIANELLI, IGNACIO ; CASTELLO, JORGE P. ; VERA, CAROLINA ; BURATTI, CLAUDIO ; PÁJARO, MARCELO ; Pezzi, Luciano P. ; MÖLLER, OSMAR O. . 
  Climate change impacts on the atmospheric circulation, ocean, and fisheries in the southwest South Atlantic Ocean: a review. CLIMATIC CHANGE, v. 162, p. 2359-2377, 
  doi:\textcolor{bleu_cite}{\href{https://doi.org/10.1007/s10584-020-02783-6}{10.1007/s10584-020-02783-6}}, 2020.

  \item[] ROSA, ELIANA BERTOL ; PEZZI, LUCIANO PONZI ; QUADRO, MARIO FRANCISCO LEAL DE ; BRUNSELL, 
  NATHANIEL . Automated Detection Algorithm for SACZ, Oceanic SACZ, and Their Climatological Features. 
  FRONTIERS IN ENVIRONMENTAL SCIENCE, v. 8, p. 1, 
  doi:\textcolor{bleu_cite}{\href{http://dx.doi.org/10.3389/fenvs.2020.00018}{10.3389/fenvs.2020.00018}}, 2020. 

  \item[] SWART, S. ; GILLE, S. T. ; DELILLE, B. ; JOSEY, S. ; MAZLOFF, M. ; NEWMAN, L. ; THOMPSON, A. F. ; 
  THOMSON, J. ; WARD, B. ; PLESSIS, M. D. ; KENT, E. C. ; GIRTON, J. ; GREGOR, L. ; H, P. ; HYDER, P. ; 
  PEZZI, L. P. ; SOUZA, R. B. ; TAMSITT, V. ; WELLER, R. A. ; ZAPPA, C. J. . Constraining Southern 
  Ocean air-sea-ice fluxes through enhanced observations. FRONTIERS IN MARINE SCIENCE, v. 6, p. ,
  doi:\textcolor{bleu_cite}{\href{https://doi.org/10.3389/fmars.2019.00421}{10.3389/fmars.2019.00421}}, 2019. 

  \item[] SUTIL, U. A. ; PEZZI, L. P. ; ALVES, R. C. M. ; NUNES, A. B. . 
  Ocean-Atmosphere Interactions in an Extratropical Cyclone in the Southwest Atlantic. 
  ANUÁRIO DO INSTITUTO DE GEOCIÊNCIAS (UFRJ. IMPRESSO), v. 42, p. 525-535, 
  2019. 

  \item[] OLIVEIRA, R.R. ; PEZZI, L.P. ; SOUZA, R.B. ; SANTINI, M.F. ; CUNHA, L.C. ; PACHECO, F.S. . 
  First measurements of the ocean-atmosphere CO2 fluxes at the Cabo Frio upwelling system region, 
  Southwestern Atlantic Ocean. CONTINENTAL SHELF RESEARCH, v. 181, p. 135-142,
  doi:\textcolor{bleu_cite}{\href{https://doi.org/10.1016/j.csr.2019.05.008}{10.1016/j.csr.2019.05.008}}, 2019. 
   

  \item[] ENDO, CLARISSA AKEMI KAJIYA ; GHERARDI, DOUGLAS FRANCISCO MARCOLINO ; Pezzi, Luciano Ponzi 
  ; LIMA, LEONARDO NASCIMENTO . Low connectivity compromises the conservation of reef fishes by 
  marine protected areas in the tropical South Atlantic. Scientific Reports, v. 9, p. 8634, 
  2019.

  \item[] QUADRO, MARIO ; ROSA, E. B. ; PEZZI, L. P. . O Climanálise e o monitoramento da ZCAS nos 
  últimos 30 anos. Climanalise (São José dos Campos), v. 1, p. 1, 2018. 

  \item[] DE DECCO, HATSUE TAKANACA ; TORRES JUNIOR, AUDALIO REBELO ; Pezzi, Luciano Ponzi ; 
  LANDAU, LUIZ . Revisiting tropical instability wave variability in the Atlantic ocean using 
  SODA reanalysis. Ocean Dynamics, v. 68, p. 327-345, 
  2018.

  \item[] HACKEROTT, J. A. ; PEZZI, L. P. ; BAKHODAY PASKYABI, M. ; OLIVEIRA, A. P. ; REUDER, J. ; 
  DE SOUZA, R. B. ; DE CAMARGO, R. . The Role of Roughness and Stability on the Momentum Flux in 
  the Marine Atmospheric Surface Layer: A Study on the Southwestern Atlantic Ocean. JOURNAL OF 
  GEOPHYSICAL RESEARCH-ATMOSPHERES, v. 123, p. 3914-3932, 2018. 
  2018.

  \item[] LIMA, L. N. ; PEZZI, L. P. ; PENNY, S. G. ; TANAJURA, C. A. S. . An investigation of ocean 
  model uncertainties through ensemble forecast experiments in the Southwest Atlantic Ocean. 
  JOURNAL OF GEOPHYSICAL RESEARCH-OCEANS, v. 124, p. 432,
  2018.

  \item[] PACHECO, FELIPE S. ; MIRANDA, MARCELA ; PEZZI, L. P. ; ASSIREU, ARCILAN ; MARINHO, 
  MARCELO M. ; MALAFAIA, MÁRCIO ; REIS, ANDRÉ ; SALES, MATIAS ; CORREIA, GILSINÉIA ; DOMINGOS, 
  PATRÍCIA ; IWAMA, ALLAN ; RUDORFF, CONRADO ; OLIVA, PEDRO ; OMETTO, JEAN P. . Water quality 
  longitudinal profile of the Paraíba do Sul River, Brazil during an extreme drought event. 
  LIMNOLOGY AND OCEANOGRAPHY, v. 1, p. 1,
  2017.

  \item[] Mendonça, L. F. ; SOUZA, R. B. ; ASEFF, C. R. C. ; PEZZI, L. P. ; MÖLLER, O. O. ; 
  ALVES, R. C. M. . Regional modeling of the water masses and circulation annual variability 
  at the Southern Brazilian Continental Shelf. JOURNAL OF GEOPHYSICAL RESEARCH-OCEANS, v. 
  122, p. 1232-1253,
  2017.

  \item[] MILLER, ARTHUR J. ; COLLINS, MAT ; GUALDI, SILVIO ; JENSEN, TOMMY G. ; MISRA, VASU ; PEZZI, 
  L. P. ; PIERCE, DAVID W. ; PUTRASAHAN, DIAN ; SEO, HYODAE ; TSENG, YU-HENG . 
  Coupled ocean-atmosphere modeling and predictions. JOURNAL OF MARINE RESEARCH, v. 75, p. 
  361-402,
  2017.

  \item[] PULLEN, J. ; ALLARD, R. ; SEO, HYODAE ; Miller, A. J. ; CHEN, S. ; PEZZI, L. P. ; SMITH, T. ; 
  CHU, P. ; ALVES, J. ; CALDERA, R. . Coupled ocean-atmosphere forecasting at short and medium time 
  scales. JOURNAL OF MARINE RESEARCH, v. 75, p. 877-921,
  2017.

  \item[] PEZZI, L. P.; SOUZA, R. B. ; QUADRO, M. . Uma revisão dos processos de interação 
  oceano-atmosfera em regiões de intenso gradiente termal do Oceano Atlântico Sul baseada em dados 
  observacionais. Revista Brasileira de Meteorologia (Impresso), v. 31, p. 1,
  2016.

  \item[] MACHADO, J. P. ; JUSTINO, F. ; PEZZI, L. P. . Impacts of Wind Stress 
  Changes on the Global Heat Transport, Baroclinic Instability, and the Thermohaline Circulation. 
  Advances in Meteorology, v. 2016, p. 1-15,
  2016.

  \item[] PACHECO, FELIPE S. ; MIRANDA, MARCELA ; OMETTO, JEAN P. ; ASSIREU, ARCILAN ; PEZZI, LUCIANO . 
  Waterfall Project: Sport, Science and Society Coming Together. Limnology and Oceanography Bulletin, 
  v. 25, p. 1-6, 2016.

  \item[] PEZZI, L. P.; SOUZA, R. B. ; FARIAS, P. C. ; ACEVEDO, O. ; MILLER, A. J. . Air-sea 
  interaction at the Southern Brazilian Continental Shelf: In situ observations. JOURNAL OF 
  GEOPHYSICAL RESEARCH-OCEANS, v. 121, p. 6671-6695.

  \item[] FARIAS, P. ; SOUZA, R. B. ; PEZZI, L. P. ; DIAS, F. ; ROSSATO, F. ; ALVES, R. C. M. . 
  ANÁLISE DO ACOPLAMENTO OCEANO-ATMOSFERA EM ESCALA SINÓTICA AO LONGO DE 33 °S NOS DIAS 19 E 20 DE 
  JUNHO DE 2012. CIÊNCIA E NATURA, v. 37, p. 39, 2016.

  \item[] D'AGOSTINI, ANDRESSA ; GHERARDI, DOUGLAS FRANCISCO MARCOLINO ; Pezzi, Luciano Ponzi . 
  Connectivity of Marine Protected Areas and Its Relation with Total Kinetic Energy. Plos One, v. 10, 
  p. e0139601, 2015.

  \item[] PARISE, CLAUDIA K. ; PEZZI, Luciano P. ; HODGES, KEVIN I. ; Justino, Flavio . The Influence 
  of Sea Ice Dynamics on the Climate Sensitivity and Memory to Increased Antarctic Sea Ice. 
  Journal of Climate, v. 28, p. 150921125611001,
  2015.

  \item[] SILVEIRA, ISABEL PORTO ; PEZZI, Luciano Ponzi . Sea surface temperature anomalies driven by
  oceanic local forcing in the Brazil-Malvinas Confluence. Ocean Dynamics (Print), v. 64, p. 347-360, 
  2014.

  \item[] SOARES, HELENA CACHANHUK ; GHERARDI, DOUGLAS FRANCISCO MARCOLINO ; Pezzi, Luciano Ponzi ; 
  Kayano, Mary Toshie ; PAES, EDUARDO TAVARES . Patterns of interannual climate variability in Large 
  Marine Ecosystems. Journal of Marine Systems, v. 134, p. 57-68,
  2014.

  \item[] DIAS, DANIELA FAGGIANI ; Pezzi, Luciano Ponzi ; GHERARDI, DOUGLAS FRANCISCO MARCOLINO ; 
  CAMARGO, RICARDO . Modeling the Spawning Strategies and Larval Survival of the Brazilian Sardine 
  (Sardinella brasiliensis). Progress in Oceanography, v. 123, p. 38-53,
  2014.

  \item[] MACHADO, J.P. ; JUSTINO, FLÁVIO ; PEZZI, L. P. . Efeitos do aumento da tensão de cisalhamento 
  do vento no clima do Hemisfério Sul obtido do modelo acoplado SPEEDO. Revista Brasileira de 
  Meteorologia (Impresso), v. 29, p. 597-612,
  2014.

  \item[] DE CAMARGO, RICARDO ; TODESCO, ENZO ; Pezzi, Luciano Ponzi ; DE SOUZA, RONALD BUSS . 
  Modulation mechanisms of marine atmospheric boundary layer at the Brazil-Malvinas Confluence region. 
  JOURNAL OF GEOPHYSICAL RESEARCH-ATMOSPHERES, v. 118, p. 6266-6280, 2013.

  \item[] MUNCHOW, G. B. ; ALVES, R. C. M. ; PEZZI, Luciano P. . EFEITO DO ACOPLAMENTO DOS MODELOS 
  NUMÉRICOS ROMS-WRF NA CAMADA LIMITE PLANETÁRIA - UM ESTUDO DE CASO SOBRE A CONFLUÊNCIA 
  BRASIL-MALVIAS. CIÊNCIA E NATURA, v. 0, p. 4,2013.

  \item[] SOARES, H. C. ; PEZZI, L. P. ; GHERARDI, Douglas F. M. ; PAES, E. T. . Oceanic and 
  atmospheric patterns during spawning periods prior to extreme catches of the Brazilian sardine 
  (Sardinella brasiliensis) in the southwest Atlantic. Scientia Marina, v. 75, p. 665-677,
  2011.

  \item[] Soppa, Mariana Altenburg ; SOUZA, Ronald Buss de ; Pezzi, Luciano Ponzi . Variabilidade das 
  anomalias de temperatura da superfície do mar no oceano atlântico sudoeste e sua relação com o 
  fenomeno El Niño-Oscilação Sul. Revista Brasileira de Meteorologia (Impresso), v. 26, p. 375-391, 
  2011.

  \item[] Machado, Jeferson Prietsch ; Justino, Flavio ; Pezzi, Luciano Ponzi . Changes in the global 
  heat transport and eddy-mean flow interaction associated with weaker thermohaline circulation. 
  International Journal of Climatology, v. n/a, p. n/a-n/a, 2011.

  \item[] SILVEIRA, I. P. ; PEZZI, L. P. ; Souza, Ronald B. . Dois casos de ATSM analisados através de
  balanço de calor parcial para o Atlântico Sudoeste. CIÊNCIA E NATURA, v. 1, p. 375-377,
  2011.

  \item[] 
  GHERARDI, Douglas F. M. ; PAES, E. T. ; SOARES, H. C. ; PEZZI, L. P. ; KAYANO, Mary . 
  Differences between spatial patterns of climate variability and large marine ecosystems in 
  the western South Atlantic. Pan-American Journal of Aquatic Sciences, v. 5, p. 310-319, 2010.

  \item[] Acevedo, Otávio C. ; Pezzi, Luciano P. ; Souza, Ronald B. ; Anabor, Vagner ; Degrazia, 
  Gervásio A. . Atmospheric boundary layer adjustment to the synoptic cycle at the Brazil-Malvinas 
  Confluence, South Atlantic Ocean. Journal of Geophysical Research, v. 115, p. D22107,
  2010.

  \item[] GOMES JUNIOR, J. G. ; PEZZI, L. P. . O El Niño Modoki e seu impacto sobre a seca de 2005 na 
  Amazônia. ANAIS HIDROGRÁFICOS, v. 67, p. 89-98, 2010. 

  \item[] Pezzi, Luciano Ponzi; Acevedo, Otávio ; Wainer, Ilana ; Mata, Mauricio M. . Multiyear 
  measurements of the oceanic and atmospheric boundary layers at the Brazil-Malvinas confluence region.
  Journal of Geophysical Research, v. 114, p. D19103, 2009.

  \item[] Pezzi, L. P.; Mendonça, Antônio M. . A sensitivity study using two different convection 
  schemes over south america. Revista Brasileira de Meteorologia (Impresso), v. 23, p. 170-190, 
  2008.

  \item[] Pezzi, L. P. ; Kayano, Mary T. . Esquema Estatístico de Combinação e Correção de Previsões 
  Climáticas - ECCOCLIM. Revista Brasileira de Meteorologia (Impresso), v. 23, p. 347-359,
  2008.

  \item[] Pezzi, L. P.; Kayano, Mary Toshie . An analysis of the seasonal precipitation forecasts in 
  South America using wavelets. International Journal of Climatology, v. 28, p. n/a-n/a, 2008.

  \item[] Pezzi, L. P.; CALTABIANO, A. ; CHALLENOR, P. . Satellite observations of the Pacific tropical 
  instability wave characteristics and their interannual variability. International Journal of 
  Remote Sensing (Print), v. 27, n.8, p. 1581-1599,
  2006.

  \item[] PEZZI, L. P.; CALTABIANO, Antonio C ; ROBINSON, Ian . Multi-year satellite observations of 
  instability waves in the Tropical Atlantic Ocean. Ocean Science, Grã-Bretanha, v. 1, n.2, p. 
  97-112, 2005.

  \item[] GIAROLLA, Emanuel ; NOBRE, P. ; MALAGUTTI, Marta ; PEZZI, L. P. . The Atlantic Equatorial 
  Undercurrent: PIRATA observations and simulations with GFDL Modular Ocean Model at CPTEC. 
  Geophysical Research Letters, EUA, v. 32, n.10, p. L10617, 2005.

  \item[] Pezzi, L. P.; Dourado, M. S. ; Mata, M. M. ; Silva-Dias, M. A. F. . Ocean-atmosphere in situ 
  observations at the Brazil-Malvinas Confluence region. Geophysical Research Letters, EUA, v. 32, 
  n.22, p. L22603,2005.

  \item[] PEZZI, L. P.; VIALARD, Jerome ; RICHARDS, Kelvin J ; MENKES, Christophe ; ANDERSON, David . 
  Influence of ocean-atmosphere coupling on the properties of Tropical Instability Waves. 
  Geophysical Research Letters, v. 31, n.L16306, p. L16306,, 2004.

  \item[] PEZZI, L. P.; RICHARDS, Kelvin J . The effects of lateral mixing on the mean state and eddy 
  activity of an equatorial ocean. Journal of Geophysical Research, EUA, v. 108, n.C12, p. 3371, 2003.

  \item[] PEZZI, L. P.; Marengo, J. A. ; CAVALCANTI, I. F. A. ; SATYAMURTY, P. ; Trosnikov, I. ; 
  Nobre, C. A. ; CAMARGO, H. ; Sanches, M. B. ; Castro, C. A. C. ; D'Almeida, C. ; Candido, L. . 
  Assessment of regional seasonal rainfall predictability using the CPTEC/COLA atmospheric GCM. 
  Climate Dynamics, v. 21, p. 459-475,
  2003.

  \item[] 	CAVALCANTI, I. F. A. ; MARENGO, J ; SATYAMURTY, P. ; TROSNIKOV, I ; TARASOVA, T ; CASTRO, 
  C A ; SANCHES, M ; CAMARGO, H. ; PEZZI, L. P. . Global climatological features in a simulation 
  using the CPTEC-COLA AGCM. Journal of Climate, EUA, v. 15, n.21, p. 2965-2988,

  \item[] PEZZI, L. P.; CAVALCANTI, I. F. A. . The Relative Importance of ENSO and Tropical 
  Atlantic Sea Surface temperature anomalies for seasonal precipitation over South America. 
  Climate Dynamics, v. 17, p. 205-212, 2001. 

  \item[] CAVALCANTI, I. F. A. ; SATYAMURTY, Prakki ; MARENGO, Jose Antonio ; NOBRE, Carlos A ;
  TROSNIKOV, Igor ; MANZI, Antonio Oncimar ; TARASOVA, Tatiana ; D'ALMEIDA, Cassiano ; SAMPAIO, 
  Gilvan ; CASTRO, Christopher C ; SANCHES, Marcos ; CAMARGO, Hellio ; PEZZI, L. P. . Climate 
  Characteristics in an ensemble simulation using the CPTEC/COLA atmospheric global circulation 
  model. Instituto Nacional de Pesquisas Espaciais Inpe, INPE, v. INPE, n.8150, p. 1-71, 2001. 

  \item[] KAYANO, M. ; TOTA, J. ; PEZZI, L. P. . On the influence of the El Niño, La Niña and atlantic
  dipole pattern on the amazonian rainfall during 1960-1998.. ACTA AMAZONICA, v. 30, n.2, p. 
  305-318, 2000. 

  \item[] PEZZI, L. P.; Ubarana, V. ; REPELLI, C. . Desempenho e Previsões de um Modelo Regional 
  Estatístico para a Região Sul do Brasil. Revista Brasileira de Geofísica (Impresso), v. 18, 
  p. 129, 2000. 

  \item[] CAVALCANTI, I. F. A. ; PEZZI, L. P. ; SANCHES, M . Climate prediction of precipitation 
  over South America for MAM 1999. Experimental Long Lead Forecast Bulletin, v. 8, n.1, p. 
  51-54, 1999. 

  \item[] SATYAMURTY, P. ; PEZZI, L. P. . Possible teleconnections of winter rainfall in southern 
  Brazil with Indian monsoon activity. Meteorology and Atmospheric Physics, v. 68, n.1-2, p. 
  53-56, 1998. 

  \item[] PEZZI, L. P.; CAVALCANTI, I. F. A. . Temperature and precipitation anomalies over Brazil
  during the 1995 winter season: atmospheric and oceanic characteristics. Revista Brasileira de 
  Geofísica (Impresso), v. 16, n.2-3, p. 209-218, 1998. 
  
  \item[] KAYANO, M. T. ; PEZZI, L. P. . El Niño e o clima no Brasil. Laranja \& Cia, Matão - SP, v. 
  47, p. 6-6, 1998. 

  \item[] CAVALCANTI, I. F. A. ; PEZZI, L. P. ; NOBRE, P. ; CAMARGO, H. . Climate Prediction In Brazil 
  For The Nordeste Rainy Season (Mam) 1998. Experimental Long Lead Forecast Bulletin, Maryland - 
  USA, v. 07, n.1, p. 24-27, 1998. 

  \item[] PEZZI, L. P.; REPELLI, C. A. ; NOBRE, P. ; CAVALCANTI, I. F. A. . Forecasts of Tropical 
  Atlantic SST anomalies using a statistical ocean model at CPTEC/INPE-Brazil, 1998. Experimental 
  Long Lead Forecast Bulletin, Maryland - USA, v. 7, n.1, p. 28-31, 1998. 

  \item[] 
  CAVALCANTI, I. F. A. ; PEZZI, L. P. ; SANCHES, M . Climate prediction of precipitation over South 
  America for DJF and MAM 1999.. Experimental Long Lead Forecast Bulletin, v. 7, n.4, p. 24-27, 
  1998. 

  \item[] PEZZI, L. P.; ARAVÉQUIA, J. A. ; UVO, C. B. ; HERDIES, D. L . 
  Perspectiva para a estação chuvosa no norte do Nordeste em 1994. Climanálise (São José dos Campos), 
  São José dos Campos - SP, v. 9, n.5, 1994. 

  \item[] UVO, C. B. ; ARAVÉQUIA, J. A. ; PEZZI, L. P. ; HERDIES, D. L . 
  Perspectiva para a estação chuvosa no norte do Nordeste em 1994. Climanálise (São José dos Campos), 
  São José dos Campos - SP, v. 9, n.6, 1994. 

  \item[] UVO, C. B. ; ARAVÉQUIA, J. A.. ; HERDIES, Dirceu L ; PEZZI, L. P. . 
  Perspectivas para a estação chuvosa no norte do Nordeste em 1994. Climanálise (São José dos Campos), 
  São José dos Campos - SP, v. 9, n.3, 1994. 

  \item[] 
  PEZZI, L. P.; HERDIES, Dirceu L ; ARAVÉQUIA, José Antonio ; MOURA, Antonio Divino . 
  Perspectivas para a estação chuvosa no norte do Nordeste em 1994. Climanálise (São José dos Campos), 
  São José dos Campos - SP, v. 9, n.2, 1994. 

  \item[] 
  HERDIES, D. L. ; PEZZI, L. P. ; ARAVÉQUIA, J. A. ; NOBRE, P. . Perspectivas para a 
  estação chuvosa no norte do Nordeste em 1994. Climanálise (São José dos Campos), São José dos Campos, 
   9, n.1, 1994. 

  \item[] 
  ARAVÉQUIA, J. A. ; PEZZI, L. P. ; UVO, C. B. ; HERDIES, D. L. . Perspectivas 
  para a estação chuvosa no norte do Nordeste em 1994. Climanálise (São José dos Campos), 
  São José dos Campos - SP, v. 9, n.4, 1994. 
\end{itemize}

\lettersection{Capítulos de livros publicados}

\begin{itemize}
  \item[] LIMA, L. N. ; PEZZI, L. P. . Assimilação de dados com o método LETKF no Oceano Atlântico 
  Sudoeste: A importância das observações de satélite. In: Leonardo Tulio. (Org.). Aplicações e
  Princípios do Sensoriamento Remoto 2.. 1ed.Ponta Grossa, Paraná: Atena, 2018, v. 2, p. 54-68. 

  \item[] PEZZI, L. P.. PBMC, 2012: Sumario Executivo do Volume 1. In: Moacyr Araujo; 
  Tércio Ambrizzi. (Org.). Base Cientifica das Mudancas Climaticas. Contribuicao do Grupo de 
  Trabalho 1 para o 1 Relatório de Avaliação Nacional do Painel Brasileiro de Mudancas Climaticas. 
  1ed.: , 2012, v. 1, p. 1-34. 

  \item[] GARCIA, C. A. E. ; MATA, M. M. ; GARCIA, V. M. T. ; WAINER, I. E. K. ; 
  ITO, R. G. ; SOUZA, R. B. ; PEZZI, L. P. ; POLLERY, R. ; EVANGELISTA, H. . Estudos no 
  Oceano Austral para a compreensão do clima global. In: Ministério da Ciência e Tecnologia. 
  (Org.). Ciência brasileira no IV Ano Polar Internacional. 1ed.Brasilia - DF: Ministério da 
  Ciência e Tecnologia, 2009, v. , p. 99-109. 

  \item[] PEZZI, L. P.; SOUZA, R. B. . Variabilidade de meso-escala e 
  interação Oceano-Atmosfera no Atlântico Sudoeste. In: Iracema F. A. Cavalcanti; 
  Nelson J. Ferreira; Maria Assunção F. Dias; Maria Gertrudes A. Justi. (Org.). 
  Tempo e Clima no Brasil. 1ed.São Paulo: Oficina de Textos, 2009, v. 1, p. 385-405. 

  \item[] 
  GARCIA, Carlos A E ; MATA, Mauricio M ; GARCIA, Virginia M T ; KINAS, P. G. ; SECCHI, E. R. ; 
  MUELBERT, M. M. ; SOUZA, R. B. ; PEZZI, L. P. ; POLLERY, R. ; KURTZ, F . 
  Oceano Austral. In: Tania Brito. (Org.). O Brasil e o meio ambiente antártico. Brasilia: 
  Ministério do Meio Ambiente, 2007, v. 1, p. 41-49.

  \item[] PEZZI, L. P.; SOUZA, E. B. . O Uso da Temperatura da Superfície do Mar 
  em Estudos Climáticos. In: Ronald Buss de Souza. (Org.). Oceanografia por Satélites. 
  1ed.São Paulo: Oficina de Textos, 2005, v. 1, p. 117-133. 

  \item[] PEZZI, L. P.; ROSA, M. B. ; BATISTA, N. N. M. . A Corrente de Jato Subtropical na 
  América do Sul. In: Climanalise. (Org.). Climanálise Especial - Edição Comemorativa de 
  10 Anos. 1ed.Cachoeria Paulista - SP: CPTEC/INPE - ISSN 0103-0019, 1996, v. 1, p. -. 
\end{itemize}

\lettersection{Supervisão de pós-doutorado}
\begin{itemize}
  \item[] Marcelo Freitas Santini. 2018. Instituto Nacional de Pesquisas Espaciais, Coordenação de Aperfeiçoamento de Pessoal de Nível Superior. Luciano Ponzi Pezzi. 
  \item[] Isabel porto da silveira. 2016. Instituto Nacional de Pesquisas Espaciais, . Luciano Ponzi Pezzi. 
  \item[] Ricardo de Camargo. Aplicações do Método de Assimilação LETKF à Modelagem da Circulação Oceânica no Atlântico Sul: Representação da evolução das anomalias de TSM. 2009. Instituto Nacional de Pesquisas Espaciais, Fundação de Amparo à Pesquisa do Estado de São Paulo. Luciano Ponzi Pezzi. 
\end{itemize}

\lettersection{Teses de doutorado}
\begin{itemize}
  \item[] Celina Cândida Ferreira Rodrigues. Estudos de fluxos turbulentos na interface oceano-atmosfera em altas latitudes. Início: 2018. Tese (Doutorado em Sensoriamento Remoto) - Instituto Nacional de Pesquisas Espaciais, Coordenação de Aperfeiçoamento de Pessoal de Nível Superior. (Orientador). 
  \item[] Mylene Jaen Cabrera. Interação oceano-atmosfera em altas latitudes. Início: 2018. Tese (Doutorado em Meteorologia) - Instituto Nacional de Pesquisas Espaciais, Conselho Nacional de Desenvolvimento Científico e Tecnológico. (Orientador). 
  \item[] Eliane Bertol Rosa. Detecção automática e estudo da influencia oceânica na Zona de Convergência do Atlântico Sul. Início: 2017. Tese (Doutorado em Sensoriamento Remoto) - Instituto Nacional de Pesquisas Espaciais, Coordenação de Aperfeiçoamento de Pessoal de Nível Superior. (Orientador). 
  \item[] Leonardo Nascimento de Lima. Estudo das incertezas na simulação por conjunto e no uso da assimilação de dados no oceano Atlântico Sudoeste.. 2018. Tese (Doutorado em Meteorologia) - Instituto Nacional de Pesquisas Espaciais, Coordenação de Aperfeiçoamento de Pessoal de Nível Superior. Orientador: Luciano Ponzi Pezzi. 
  \item[] Regiane Moura. Mecanismos de estabilidade da Camada Limite Atmosférica Marinha na região da Confluência Brasil-Malvinas. 2017. Tese (Doutorado em Meteorologia) - Instituto Nacional de Pesquisas Espaciais, Coordenação de Aperfeiçoamento de Pessoal de Nível Superior. Orientador: Luciano Ponzi Pezzi. 
  \item[] João Hackerott. Momentum fluxes in the Marine Atmospheric Surface Layer: A study on the Southwestern Atlantic Ocean.. 2017. Tese (Doutorado em Meteorologia) - Universidade de São Paulo, Coordenação de Aperfeiçoamento de Pessoal de Nível Superior. Coorientador: Luciano Ponzi Pezzi.
  \item[] Marcelo Freitas Santini. Determinação dos fluxos turbulentos de calor e momentum entre o oceano e a atmosfera na região do Oceano Atlântico Sudoeste.. 2017. Tese (Doutorado em Doutorado em Meteorologia) - Universidade Federal de Santa Maria, Coordenação de Aperfeiçoamento de Pessoal de Nível Superior. Coorientador: Luciano Ponzi Pezzi. 
  \item[] Isabel Porto da Silveira. O papel do conteúdo de calor oceânico na manutenção de anomalias de temperatura da superfície do mar do Atlântico Sul. 2014. Tese (Doutorado em Meteorologia) - Instituto Nacional de Pesquisas Espaciais, Coordenação de Aperfeiçoamento de Pessoal de Nível Superior. Orientador: Luciano Ponzi Pezzi. 
  \item[] Claudia Klose Parise. Sensitivity and memory of the current mean climate to increased Antarctic sea-ice: The role of sea ice dynamics. 2014. Tese (Doutorado em Meteorologia) - Instituto Nacional de Pesquisas Espaciais, Conselho Nacional de Desenvolvimento Científico e Tecnológico. Orientador: Luciano Ponzi Pezzi. 
  \item[] Helena Cachanhuk Soares. Variabilidade climática interanual local e remota do Atlântico sobre os Grandes Ecossistemas Marinhos Brasileiros. 2014. Tese (Doutorado em Sensoriamento Remoto) - Instituto Nacional de Pesquisas Espaciais, Coordenação de Aperfeiçoamento de Pessoal de Nível Superior. Coorientador: Luciano Ponzi Pezzi. 
  \item[] Jeferson Prietsch Machado. Impactos das circulações oceânicas e atmosféricas associada a variabilidade na tensão de cisalhamentodo vento. 2013. Tese (Doutorado em Agronomia (Meteorologia Aplicada)) - Universidade Federal de Viçosa, Coordenação de Aperfeiçoamento de Pessoal de Nível Superior. Coorientador: Luciano Ponzi Pezzi. 
\end{itemize}

\lettersection{Dissertações de mestrado}
\begin{itemize}
  \item[] Giullian Nícola Lima dos Reis. Estudo das ondas de instabilidade tropical do Atlântico e seus impactos na atmosfera. Início: 2018. Dissertação (Mestrado em Sensoriamento Remoto) - Instituto Nacional de Pesquisas Espaciais, Coordenação de Aperfeiçoamento de Pessoal de Nível Superior. (Orientador). 
  \item[] Manuel Agostinho Victor Antonio. RELEVÂNCIA DA ASSIMILAÇÃO DE RADIOSSONDAGENS NO ESTUDO DE PROCESSOS DE INTERAÇÃO OCEANO-ATMOSFERA. Início: 2018. Dissertação (Mestrado em Meteorologia) - Instituto Nacional de Pesquisas Espaciais, Coordenação de Aperfeiçoamento de Pessoal de Nível Superior. (Orientador). 
  \item[] Raquel Reno de Oliveira. Análise temporal da interação Oceano-Atmosfera no Atlantico Sudoeste. 2018. Dissertação (Mestrado em Sensoriamento Remoto) - Instituto Nacional de Pesquisas Espaciais, Coordenação de Aperfeiçoamento de Pessoal de Nível Superior. Orientador: Luciano Ponzi Pezzi. 
  \item[] Mylene Jean Cabrera. Um estudo numérico dos processos de Interação Oceano-Atmosfera em frentes oceanicas. 2018. Dissertação (Mestrado em Meteorologia) - Instituto Nacional de Pesquisas Espaciais, Coordenação de Aperfeiçoamento de Pessoal de Nível Superior. Orientador: Luciano Ponzi Pezzi. 
  \item[] Clarissa Akemi Kajiya Endo. DETERMINAÇÃO DA CONECTIVIDADE ECOLÓGICA ENTRE AS ILHAS OCEÂNICAS BRASILEIRAS E A PLATAFORMA CONTINENTAL NORTE E LESTE DO BRASIL. 2018. Dissertação (Mestrado em Sensoriamento Remoto) - Instituto Nacional de Pesquisas Espaciais, Coordenação de Aperfeiçoamento de Pessoal de Nível Superior. Coorientador: Luciano Ponzi Pezzi. 
  \item[] Carlos Diego Gurjão. Influencia conjunta do El Niño Oscilação Sul e Dipolo da Antártica no gelo marinho doa mares de Bellingshaussen-Amundsen e mar de Weddell. 2017. Dissertação (Mestrado em Meteorologia) - Instituto Nacional de Pesquisas Espaciais, Conselho Nacional de Desenvolvimento Científico e Tecnológico. Orientador: Luciano Ponzi Pezzi. 
  \item[] Eliana Bertol Rosa. Identificação de episódios de ZCAS utilizando um método automático de classificação de imagens orbitais de ROLE. 2017. Dissertação (Mestrado em Sensoriamento Remoto) - Instituto Nacional de Pesquisas Espaciais, Conselho Nacional de Desenvolvimento Científico e Tecnológico. Orientador: Luciano Ponzi Pezzi. 
  \item[] Ueslei Adriano Sutil. Estudo do Ciclone Extratropical no Atlântico Sudoeste: Uma abordagem numérica. 2016. Dissertação (Mestrado em Sensoriamento Remoto) - Universidade Federal do Rio Grande do Sul, Coordenação de Aperfeiçoamento de Pessoal de Nível Superior. Coorientador: Luciano Ponzi Pezzi. 
  \item[] Priscila Cavalheiro Farias. Fluxos de calor e dióxido de carbono entre o oceano e a atmosfera na região costeira e oceânica ao sul do Brasil. 2014. Dissertação (Mestrado em Meteorologia) - Universidade Federal de Santa Maria, . Coorientador: Luciano Ponzi Pezzi. 
  \item[] Andressa D'Agostini da Silva. Determinação da Conectividade de Ambientes Recifais utilizando Modelagem Biofísica. 2014. Dissertação (Mestrado em Sensoriamento Remoto) - Instituto Nacional de Pesquisas Espaciais, Coordenação de Aperfeiçoamento de Pessoal de Nível Superior. Coorientador: Luciano Ponzi Pezzi. 
  \item[] Daniela Faggiani Dias. Determinação dos padrões de desova e da sobrevivência das larvas da Sardinha-Verdadeira (Sardinela Brasiliensis) na Plataforma Continental Sudeste do Brasil, utilizando baseada em individuo. 2013. Dissertação (Mestrado em Sensoriamento Remoto) - Instituto Nacional de Pesquisas Espaciais, Coordenação de Aperfeiçoamento de Pessoal de Nível Superior. Coorientador: Luciano Ponzi Pezzi. 
  \item[] Cristina Schultz. Ciclos biogeoquímicos e modos de variabilidade climática na região da Confluência Brasil-Malvinas. 2012. Dissertação (Mestrado em Meteorologia) - Instituto Nacional de Pesquisas Espaciais, Coordenação de Aperfeiçoamento de Pessoal de Nível Superior. Orientador: Luciano Ponzi Pezzi. 
  \item[] Walid Maia Pinto Silva e Seba. Impacto da TSM de alta resolução em simulações numéricas da atmosfera sobrejacente a Região da Confluência Brasil-Malvinas. 2011. Dissertação (Mestrado em Meteorologia) - Instituto Nacional de Pesquisas Espaciais, Conselho Nacional de Desenvolvimento Científico e Tecnológico. Orientador: Luciano Ponzi Pezzi. 
  \item[] Isabel Porto da Silveira. Estudo das Anomalias Atmosféricas e Oceânicas na Região da Confluência Brasil-Malvinas. 2010. Dissertação (Mestrado em Meteorologia) - Instituto Nacional de Pesquisas Espaciais, Fundação de Amparo à Pesquisa do Estado de São Paulo. Orientador: Luciano Ponzi Pezzi. 
  \item[] Lucimara Russo. Interação Oceano-Atmosfera sobre o Atlântico Sudoeste. 2009. Dissertação (Mestrado em Meteorologia) - Instituto Nacional de Pesquisas Espaciais, . Orientador: Luciano Ponzi Pezzi. 
  \item[] Jairo Geraldo Gomes Junior. Impacto de sondagens atmosféricas sobre o Atlântico Tropical no balanço de umidade da Amazônia. 2009. Dissertação (Mestrado em Meteorologia) - Instituto Nacional de Pesquisas Espaciais, . Orientador: Luciano Ponzi Pezzi. 
  \item[] Helena Cachanhuk Soares. Estudo da condições atmosféricas e oceânicas do Atlântico Sudoeste e suas associações com extremos de captura da Sardinha-Verdadeira. 2009. Dissertação (Mestrado em Meteorologia) - Instituto Nacional de Pesquisas Espaciais, Coordenação de Aperfeiçoamento de Pessoal de Nível Superior. Orientador: Luciano Ponzi Pezzi. 
  \item[] Jeferson Prietsch Machado. Resposta das Circulações Oceânica e Atmosférica Associada ao Enfraquecimento da Circulação Termohalina Global. 2009. Dissertação (Mestrado em Agronomia (Meteorologia Aplicada)) - Universidade Federal de Viçosa, Conselho Nacional de Desenvolvimento Científico e Tecnológico. Coorientador: Luciano Ponzi Pezzi. 
  \item[] Paulo Pereira Oliveira Matos. Impacto da utilização de dados de temperatura da superfície do mar de alta resolução espacial em um modelo de previsão numérica de tempo. 2009. Dissertação (Mestrado em Sensoriamento Remoto) - Instituto Nacional de Pesquisas Espaciais, . Coorientador: Luciano Ponzi Pezzi.
\end{itemize}

\lettersection{Manuais técnicos}
\begin{itemize}
  \item[] SUTIL, U. A. ; PEZZI, Luciano P. . Guia prático para utilização do COAWST - 2ª Edição. São José dos Campos: Instituto Nacional de Pesquisas Espaciais - INPE, 2019 (Relatório Técnico). 
  \item[] SUTIL, U. A. ; Pezzi, L. P. . Guia prático para utilização do COAWST. São José dos Campos: Instituto Nacional de Pesquisas Espaciais (INPE), 2018 (Relatório Técnico).
  \item[] CAVALCANTI, Iracema F A ; SATYAMURTY, Prakki ; MARENGO, J ; NOBRE, Carlos A ; TROSNIKOV, I ; TARASOVA, T ; CASTRO, C ; SANCHES, M ; CAMARGO, H ; PEZZI, L. P. . Climate characteristics in an ensemble simulation using CPTEC/COLA Atmospheric GCM. 2001 (Relatório Técnico).
  \item[] MARENGO, J ; CAVALCANTI, I. F. A. ; SATYAMURTY, P. ; TROSNIKOV, Igor ; TARASOVA, T ; D'ALMEIDA, Cassiano ; SAMPAIO, Gilvan ; CASTRO, C ; SANCHES, M ; CAMARGO, H. ; Pezzi, L. P. . Ensemble simulation of interanual climate variability using the CPTEC COLA AGCM. São José dos Campos 2001 (Relatório Técnico). 
\end{itemize}

\lettersection{Trabalhos completos publicados em anais de congressos}
\begin{itemize}
  \item[] CASAGRANDE, F. ; SOUZA, Ronald Buss de ; PEZZI, L. P. . Ocean-atmosphere coupling at the 
  Brazil-Malvinas Confluence region based on in situ, satellite and numerical model data. 
  In: 10th International Conference on Southern Hemisphere Meteorology and Oceanography, 
  2012, Noumea, New Caledonia, 2012, Noumea. Proceeding of the 10th International 
  Conference on Southern Hemisphere Meteorology and Oceanography, 2012. 
  
  \item[] PARISE, C. K. ; PEZZI, L. P. ; JUSTINO, F. B. . Memória do sistema climático acoplado a uma condição extrema de concentração de gelo marinho na Antártica. In: II Workshop Apecs-Brasil (Associação de Pesquisadores Polares em Início de Carreira), 2012, Rio Grande. Integração da Pesquisa Antártica Sul-Americana. Rio Grande: CIDEC-SUL, 2012. 
  \item[] FINOTTI, E. ; SOUZA, Ronald Buss de ; SANTINI, M. F. ; Pezzi, Luciano P. ; FARIAS, P. C. . Análise dos Fluxos de Calor Entre o Oceano e a Atmosfera ao Largo da Costa de Mostardas (RS) nos Dias 17 e 18 de Junho de 2012. In: XVII Congresso Brasileiro de Meteorlogia, 2012, Gramado. Anais do XVII CBMet, 2012. 
  \item[] ROSSATO, F. ; SOUZA, Ronald Buss de ; SANTINI, M. F. ; Pezzi, Luciano Ponzi . Fluxos de calor entre o oceano e a atmosfera acima dos contrastes horizontais termais entre as águas da Corrente do Brasil e da Corrente Costeira do Brasil. In: XVII Congresso Brasileiro de Meteorlogia, 2012, Gramado. Anais do XVII CBMet, 2012. 
  \item[] GONCALVES-ARAUJO, R. ; MENDES, C. R. B. ; ROSSATO, F. ; SOUZA, Ronald Buss de ; GARCIA, Virginia M T ; PEZZI, Luciano P. . FLUXOS DE CALOR ENTRE OCEANO-ATMOSFERA, MASSAS D?ÁGUA E FITOPLÂNCTON NO OCEANO ATLÂNTICO SUDOESTE: POSSÍVEIS IMPLICAÇÕES NOS FLUXOS DE CARBONO. In: XVII Congresso Brasileiro de Meteorologia, 2012, Gramado. Anais do XVII CBMet, 2012. 
  \item[] DIAS, D. F. ; PEZZI, Luciano P. ; GHERARDI, Douglas F. M. . CARACTERIZAÇÃO OCEÂNICA DA PLATAFORMA CONTINETAL SUDESTE DO BRASIL: RESULTADOS PRELIMINARES PARA UM ESTUDO DE DISPERSÃO DE OVOS E LARVAS DA SARDINHA-VERDADEIRA (Sardinella brasiliensis). In: XVII Congresso Brasileiro de Meteorologia, 2012, Gramado. Anais do XVII CBMet, 2012. 
  \item[] PARISE, C. K. ; PEZZI, Luciano P. . RESPOSTA DO CLIMA ATUAL A UMA CONDIÇÃO EXTREMA DE GELO MARINHO ANTÁRTICO IMPOSTA DE DIFERENTES MANEIRAS. In: XVII Congresso Brasileiro de Meteorologia, 2012, Gramado. Anais do XVII CBMet, 2012. 
  \item[] DECCO, H. T. ; PEZZI, Luciano P. ; TORRES JUNIOR, A. R. ; LANDAU, L. . TRANSPORTE DE CALOR ASSOCIADO ÀS ONDAS DE INSTABILIDADE TROPICAL NO ATLÂNTICO SUL. In: XVII Congresso Brasileiro de Meteorologia, 2012, Gramado. Anais do XVII CBMet, 2012.
  \item[] MUNCHOW, G. B. ; ABSY, J. M. ; ALVES, R. M. ; PEZZI, Luciano P. . RESULTADOS PRELIMINARES DE SIMULAÇÕES DO MODELO ACOPLADO COAWST PARA O ESTADO DO RIO GRANDE DO SUL E REGIÃO CENTRAL DO OCEANO ATLÂNTICO SUL. In: XVII Congresso Brasileiro de Meteorologia, 2012, Gramado. Anais do XVII CBMet, 2012. 
  \item[] SOARES, H. C. ; PAES, E. T. ; PEZZI, L. P. ; Kayano, Mary Toshie . Avaliação dos limites dos Grandes Ecossistemas Marinhos brasileiros em função da variabilidade climática.. In: XVI Congresso Brasileiro de Meteorologia, 2010, Belém. Anais do XVI Congresso Brasileiro de Meteorologia, 2010.
  \item[] ARSEGO, D. A. ; Souza, Ronald B. ; PEZZI, L. P. . Análise dos fluxos de calor sobre a região da Confluência Brasil-Malvinas. In: IV Congresso Brasileiro de Oceanografia, 2010, Rio Grande. Anais do IV Congresso Brasileiro de Oceanografia. Rio Grande: AOCEANO, 2010. p. 2450-2452. 
  \item[] CASAGRANDE, F. ; Soppa, Mariana Altenburg ; Pezzi, Luciano Ponzi . Análise da interação oceano-atmosfera na região da Confluência Brasil-Malvinas a partir de dados de satélite e reanálises. In: IV Congresso Brasileiro de Oceanografia, 2010, Rio Grande. Anais do IV Congresso Brasileiro de Oceanografia. Rio Grande: AOCEANO, 2010. p. 293-295. 
  \item[] PEZZI, L. P.; SILVEIRA, I. P. ; TODESCO, E. ; CASAGRANDE, F. . Processos de interação oceano-atmosfera ao longo da Frente Subtropical durante a Comissão BR-1: resultados preliminarres. In: IV Congresso Brasileiro de Oceanografia, 2010, Rio Grande. Anais do IV Congresso Brasileiro de Oceanografia. Rio Grande: AOCEANO, 2010. p. 1457-1461. 
  \item[] PEZZI, L. P. ; CASAGRANDE, F. ; ARSEGO, D. A. . Interação oceano-atmosfera sobre estruturas oceânicas de mesoescala no Oceano Atlântico Sudoeste. In: IV Congresso Brasileiro de Oceanografia, 2010, Rio Grande. Anais do IV Congresso Brasileiro de Oceanografia. Rio Grande: AOCEANO, 2010. p. 1470-1474.
  \item[] TODESCO, E. ; CAMARGO, Ricardo de ; PEZZI, L. P. . Estudo numérico da camada limite atmosférica na região da Confluência Brasil-Malvinas. In: IV Congresso Brasileiro de Oceanografia, 2010, Rio Grande. Anais do IV Congresso Brasileiro de Oceanografia. Rio Grande: AOCEANO, 2010. p. 1207-1209.
  \item[] PEZZI, L. P.; HOFFMAN, M. ; PENNY, S. . The CPTEC Ocean Data Assimilation System - CODAS. In: XV Congresso Brasileiro de Meteorologia, 2008, São Paulo. Anais do XV CBMet, 2008.
  \item[] GOMES JUNIOR, J. G. ; HERDIES, Dirceu L ; VALERIA, R. ; PEZZI, L. P. ; SAPUCCI, Luiz F ; BARBOSA, H. M. J. ; BASTARZ, C. . IMPACTO DA ASSIMILAÇÃO DE PERFIS DE ALTURA GEOPOTENCIAL PROVENIENTES DO ATOVS NO SISTEMA REGIONAL DE ASSIMILAÇÃO/PREVISÃO DO CPTEC.. In: XV Congresso Brasileiro de Meteorologia, 2008, São Paulo. Anais do XV CBMet, 2008. 
  \item[] RUSSO, L. ; PEZZI, L. P. . Análise de Observações In-Situ na Região da Confluência Brasil-Malvinas. In: XV Congresso Brasileiro de Meteorologia, 2008, São Paulo. Anasi do XV CBMet, 2008. v. 1. 
  \item[] SOARES, H. C. ; PEZZI, L. P. ; PAES, E. T. ; GHERARDI, Douglas F. M. . ESTUDO SOBRE ANOMALIAS DE TSM NO ATLÂNTICO SUDOESTE EM ANOS DE ENOS.. In: XV Congresso Brasileiro de Meteorologia, 2008, são Paulo. Anais do XVI CBMet. São Paulo, 2008. v. 1. 
  \item[] PEREIRA, A. A. ; WAINER, Ilana E. K. ; PEZZI, L. P. . Análise Comparativa da Variabilidade Climática e seu Efeito sobre o Branqueamento de Corais na Região de Abrolhos (BA, Brasil). In: XV Congresso Brasileiro de Meteorologia, 2008, São Paulo. Anais do XVI CBMet. São Paulo, 2008. v. 1.
  \item[] SOARES, H. C. ; PEZZI, L. P. ; PAES, E. T. ; GHERARDI, Douglas F. M. . Caracterização dos padrões atmosféricos e oceânicos no Atlântico Sudoeste: Associações com extremos de captura da Sardinha-verdadeira. In: XV Congresso Brasileiro de Meteorologia, 2008, São Paulo. Anais dos XV Congresso Brasileiro de Meteorologia, 2008.
  \item[] PEZZI, L. P.; SILVEIRA, I. P. ; RODRIGUES, R. R. ; GHERARDI, Douglas F. M. . Variabilidade de TSM no Atlântico Sudoeste e sua relação com o vento a superfície. In: XV Congresso Brasileiro de Meteorologia, 2008, São Paulo. Anais do XV Congresso Brasileiro de Meteorologia, 2008. 
  \item[] SOPPA, M. ; GHERARDI, Douglas F. M. ; Souza, Ronald B. ; PEZZI, L. P. . Variabilidade temporal da temperatura superficial do mar e vento estimados por satélites e reanálises em áreas de recife de coral no Brasil. In: XII Simpósio Brasileiro de Sensoriamento Remoto, 2007, Florianópolis. Anais do XII Simpósio Brasileiro de Sensoriamento Remoto. São José dos Campos: INPE, 2007. 
  \item[] GARCIA, Carlos A E ; MATA, Mauricio M ; GARCIA, Virginia M T ; SOUZA, Ronald Buss de ; MUELBERT, Monica M ; SECCHI, Eduardo ; PEZZI, L. P. ; KURTZ, F ; POLLERY, R ; ROSA, L D ; KERR, Rodrigo ; FRANCO, B . Grupo de Oceanografia de Altas Latitudes (GOAL): Principais Contribuições no Âmbito de Rde-1 de Pesquisas Antárticas. In: XIV Simpósio Brasileiro Sobre Pesquisa Antártica, 2006, São Paulo. Trabalhos Resumos do XIV Simpósio Brasileiro Sobre Pesquisa Antártica, 2006. p. 23-24. 
  \item[] PEZZI, L. P.. Does ocean-atmosphere coupling influence the properties of Tropical Instability Waves?. In: XIII Congresso Brasileiro de Meteorologia, 2004, Fortaleza. XIII Congresso Brasileiro de Meteorologia - ANAIS, 2004. 
  \item[] PEZZI, L. P.; GIAROLLA, Emanuel ; NOBRE, P. ; MALAGUTTI, Marta . A avaliação do desempenho de um OGCM forçado por um AGCM.. In: XIII Congresso Brasileiro de Meteorologia, 2004, Fortaleza. XIII Congresso Brasileiro de Meteorologia - ANAIS, 2004. 
  \item[] MARENGO, J ; CAVALCANTI, I. F. A. ; SATYAMURTY, P. ; CAMARGO, H ; CASTRO, C ; SANCHES, M ; PEZZI, L. P. . Ensemble simulation of interannual climate variability using the CPTEC/COLA Global Climate Model for the period 1982-1991. In: Sixth International Conference on Southern Hemisphere Meteorology and Oceanography., 2000, Santiago. n:Sixth International Southern Hemisphere Conference on Meteorology and Oceanography, 2000. p. 51-52. 
  \item[] PEZZI, L. P.; CAVALCANTI, I. F. A. . Precipitação sobre a América do Sul para uma situação de La Niña conjugada com dipolo positivo e negativo de TSM no Atlântico em simulações com o MCG CPTEC/COLA. In: XI Congresso Brasileiro de Agrometeorologia, 1999, Florianopolis. XI Congresso Brasileiro de Agrometeorologia, 1999. p. 1715-1719.  
  \item[] PEZZI, L. P.; CAVALCANTI, I. F. A. . Precipitação sobre a América do Sul para uma situação de El Niño conjugada com Dipolo Positivo e Negativo de TSM no Atlântico em simulações com MCG CPTEC/COLA. In: X Congresso Brasileiro de Meteorologia/VIII Congresso Latino-Americano e Ibérico de Meteorologia, 1998. Anais do X Congresso Brasileiro de Meteorologia. Brasilia - DF.  
  \item[] PEZZI, L. P.; CAVALCANTI, I. F. A. ; SERAFINI, V. ; TREUT, H. L. ; LAURENT, Z. X. . South Atlantic Convergence Zone (SACZ) in the doubling CO2 run with LMD model and ECMWF re-analysis: Part I. In: X CBMet/ VIII Congresso Latino-Americano e Ibérico de Meteorologia, 1998. Anais do X Congresso Brasileiro de Meteorologia. Brasilia - DF.
  \item[] PEZZI, L. P.; CAVALCANTI, I. F. A. ; SERAFINI, V. ; TREUT, H. L. ; LAURENT, Z. X. . South Atlantic Convergence Zone (SACZ) in the doubling CO2 run with LMD model and ECMWF re-analysis: Part II. In: X Congresso Brasileiro de Meteorologia, 1998. Anais do X CBMet/ VIII Congresso Latino-Americano e Iberico de Meteorologia. Brasilia - DF.  
  \item[] PEZZI, L. P.. Previsões regionais sazonais e comparações com observações para o Sul do Brasil durante o episódio El Niño Oscilação Sul 1997/98. In: X Congresso Brasileiro de Meteorologia, 1998. Anais X CBMet/ VIII Congresso Latino-Americano e Ibérico de Meteorologia. Brasilia - DF. 
  \item[] ROJAS, M. I. ; PEZZI, L. P. ; REPELLI, C. . Influencia de los oceanos Pacifico y Atlantico sobre el comportamiento de la precipitacion en Venezuela. In: X Congresso Brasileiro de Meteorologia, 1998. Anais do X CBMet/VIII Congresso Latino-Americano e Ibérico de Meteorologia. Brasilia - DF. 
  \item[] PEZZI, L. P.; LIVEZEY, R. ; MASUTANI, M. ; HUANG, J. ; CAVALCANTI, I. F. A. . Verificação de Previsão para a América do Sul com o GCM do NCEP, Usando Técnicas Estatísticas. In: IX Congresso Brasileiro de Meteorologia, 1996, Campos do Jordao. Anais do IX Congresso Brasileiro de Meteorologia. Campos do Jordão - SP, 1996. v. 01. p. 812-814. 
  \item[] PEZZI, L. P.; CAVALCANTI, I. F. A. . O inverno no Brasil e as anomalias identificadas na atmosfera e no oceano. In: Congresso de Meteorologia Argentino, 1996, Buenos Aires. Anais do Congresso de Meteorologia Argentino, 1996. 
  \item[] NOBRE, P. ; ABREU, M. L. ; CAVALCANTI, I. F. A. ; QUADRO, M. ; PEZZI, L. P. . Climate ensamble forecasting at CPTEC. In: Twentieth annual climate diagnostics workshop, 1995, Seattle-Washington - USA. Proceedings of the twentieth annual climate diagnostics workshop, 1995. v. 1. p. 417-420
  \item[] CAVALCANTI, I. F. A. ; NOBRE, P. ; ABREU, M. L. ; QUADRO, M. ; PEZZI, L. P. . Vertical and horizontal resolution comparisons of CPTEC/COLA GCM. In: Twentieth annual climate diagnostics workshop, 1995, Seattle-Washington - USA. Proceedings of the twentieth annual climate diagnostics workshop. Seattle - Washington - USA, 1995. v. 1. p. 73-76. 
  \item[] PEZZI, L. P.; CAVALCANTI, I. F. A. . O Jato Subtropical sobre a América do Sul no período de 1980-1989. In: VIII Congresso Brasileiro de Meteorlogia, 1994. Anais do VIII CBmet. Belo Horizonte - MG. v. 2. p. 148-151. 
  \item[] PEZZI, L. P. . Análise da Estação Chuvosa do Leste do Nordeste. In: VIII Congresso Brasileiro de Meteorologia, 1994. Anais do VIII Congresso Brasileiro de Meteorologia. Belo Horizonte - MG. v. 2. p. 155-157. 
\end{itemize}

\lettersection{Resumos expandidos publicados em anais de congressos}
\begin{itemize}
\item[] DECCO, H. T. ; PEZZI, Luciano P. ; TORRES JUNIOR, A. R. ; LANDAU, L. . Energética das Ondas de Instabilidade Tropical no Oceano Atlântico a partir das reanalises SODA. In: Congresso Brasileiro de Oceanografia, 2012, Rio de Janeiro. Anais do Congresso Brasileiro de Oceanografia. Rio de Janeiro, 2012. 

\item[] ROSSATO, F. ; SOUZA, Ronald Buss de ; PEZZI, L. P. ; CASAGRANDE, F. . Estimativas de Fluxos de calor no Oceano Atlântico Sudoeste com dados observacionais e do modelo ETA.. In: VII Workshop de Micrometeorologia, 2011, Santa Maria. Anais do VII Workshop de Micrometeorologia, 2011.

\item[] SCHULTZ, C. ; PEZZI, L. P. ; CAMARGO, Ricardo de ; ABSY, J. M. . Simulação da região da Confluência Brasil-Malvinas utilizando um modelo oceânico de alta resolução. In: IV Congresso Brasileiro de Oceanografia, 2010, Rio Grande. Anais do IVCongresso Brasileiro de Oceanografia, 2010. 

\item[] de MARIA, P. H. S. ; PEZZI, L. P. . Climatologias de Temperatura e Salinidade simuladas por diferentes versões de um Modelo Oceânico. In: IV Congresso Brasileiro de Oceanografia, 2010, Rio Grande. Anais do IV Congresso Brasileiro de Oceanografia, 2010.

\item[] SILVEIRA, I. P. ; PEZZI, L. P. . Variabilidade da CBM induzida por Forçante no Estreito de Drake. In: IV Congresso Brasileiro de Oceanografia, 2010, Rio Grande. Anais do IV Congresso Brasileiro de Oceanografia, 2010. 

\item[] PEZZI, L. P.; SOUZA, Ronald Buss de ; LENTINI, Carlos Alessandre Domingos ; GARCIA, Carlos A E ; MATA, Mauricio M . Simltaneous Ocean-Atmosphere in situ observations at the Brazil-Malvinas Confluence Region. In: 8th International Conference on Southern Hemisphere Meteorology and Oceanography, 2006, Foz do Iguacu. Proceedings of 8ICSHMO, 2006. p. 1323-1328. 

\item[] LENTINI, Carlos Alessandre Domingos ; ALMEIDA, R A F ; PEZZI, L. P. ; SOUZA, Ronald Buss de . Short-term climate variability in the southeastern South America. In: 8 th International Conference on Southern Hemisphere Meteorology and Oceanography - 8 ICSHMO, 2006, Foz do Iguaçu. Proceedings of 8th ICSHMO, 2006. p. 277-282. 

\item[] GIAROLLA, Emanuel ; NOBRE, Paulo ; PEZZI, L. P. ; MALAGUTTI, Marta . The impact of ocean initialization for SST predictions over the tropical oceans. In: 8 th International Conference on Southern Hemisphere Meteorology and Oceanography - 8 ICSHMO, 2006, Foz do Iguaçu. Proceedings of 8 th ICSHMO, 2006. p. 563-564. 
\end{itemize}

\lettersection{Resumos publicados em anais de congressos}
\begin{itemize}
\item[] SANTINI, M. F. ; SOUZA, Ronald Buss de ; PEZZI, Luciano P. . In situ measurements of the air­-sea interaction processes at the Brazil­Malvinas Confluence Region: a decade­long effort for understanding the Antarctica­ South America tele­connections.. In: Workshop on South Atlantic circulation variability and change: integrating models and observations, 2014, Buenos Aires. Proceedings of the Workshop on South Atlantic circulation variability and change: integrating models and observations. Buenos Aires, 2014. 
\item[] SOUZA, R. B. ; PEZZI, Luciano P. ; CASAGRANDE, F. ; SANTINI, M. F. . A decade of efforts on measuring the air-­sea interaction processes at the Brazil­-Malvinas Confluence Region: Why the region is a key region for understanding the Antarctica America Tele­-Connections?. In: XXXIII SCAR Bienanial Meetings and OpenScience Coference, 2014, Auckland. XXXIII SCAR Biennial Meetings, 2014. 
\item[] Machado, Jeferson Prietsch ; BLANK, D. M. P. ; JUSTINO, F. B. ; PEZZI, L. P. . Resposta das circulações oceânica e atmosférica associada às mudanças da tensão de cisalhamento do vento sobre o oceano. In: V Simpósio Internacional de Climatologia, 2013, Florianópolis. Anais V Simpósio Internacional de Climatologia, 2013. 
\item[] SCHULTZ, C. ; PEZZI, L. P. . Biogeochemical Modeling and air-sea CO2 exchange in the Southwestern Atlantic Ocean. In: Surface Ocean - Lower Atmosphere Studies (SOLAS) Summer School, 2013, Xiamen. Surface Ocean - Lower Atmosphere Studies (SOLAS), 2013.
\item[] SCHULTZ, C. ; PEZZI, L. P. ; MILLER, S. ; MARTINS, L. G. N. ; GONCALVES-ARAUJO, R. ; Acevedo, Otávio ; MOLLER JR, O. O. ; SOUZA, Ronald Buss de ; GARCIA, Virginia M T . ATLANTIC OCEAN CARBON EXPERIMENT (ACEX): IMPLEMENTATION OF EDDY COVARIANCE CO2 FLUX MEASUREMENTS ON THE SW ATLANTIC OCEAN AND RESULTS FROM THE SECOND CRUISE. In: American Geophysical Union - Meeting of the Americas, 2013, Cancun. Meeting of the Americas - AGU, 2013. 
\item[] PEZZI, L. P.; PARISE, C. K. ; SOUZA, Ronald Buss de ; GHERARDI, Douglas F. M. ; CAMARGO, Ricardo de ; SOARES, H. C. . An overview of the South Atlantic Ocean climate variability and air-sea interaction processes. In: American Geophysical Union Assembly - Meeting of the Americas, 2013, Cancun. Meeting of Americas - AGU, 2013.
\item[] ROSSATO, F. ; SOUZA, Ronald Buss de ; PEZZI, Luciano P. ; CAMPOS, E. . Análise sinótica da camada limite atmosférica no Oceano Atlântico Sudoeste entre 1 e 12 de dezembro de 2012. In: V Simpósio Internacional de Climatologia, 2013, Florianópolis. Anais do V Simpósio Internacional de Climatologia. Florianópolis, 2013.
\item[] SANTINI, M. F. ; SOUZA, Ronald Buss de ; PEZZI, Luciano P. . Análise do acopĺamento oceano­atmosfera na região da Confluência Brasil­Malvinas acima de um sistema meandrante a partir de dados observacionais. In: V Simpósio Internacional de Climatologia, 2013, Florianópolis. Anais do V Simpósio Internacional de Climatologia. Florianópolis, 2013.
\item[] FARIAS, P. ; SOUZA, R. B. ; PEZZI, Luciano P. ; DIAS, DANIELA FAGGIANI ; ROSSATO, F. . Análise do acoplamento oceano­atmosfera em escala sinótica ao longo de 33 °S no dia 19 de junho de 2012. In: V Simpósio Internacional de Climatologia, 2013, Florianópolis. Anais do V Simpósio Internacional de Climatologia. Florianópolis, 2013. 
\item[] FINOTTI, E. ; SOUZA, R. B. ; PEZZI, L. P. . Análise dos fluxos de CO2 e temperatura da superfície do mar no oceano Atlântico Sudoeste.. In: V Simpósio Internacional de Climatologia, 2013, Florianópolis. Anais do V Simpósio Internacional de Climatologia. Florianópolis, 2013. 
\item[] PEZZI, L. P.; ACEVEDO, O. ; SCHULTZ, C. . Atlantic Carbon Experiment: preliminary results from the first cruise. In: 20th Symposium on Boundary Layer Turbulence / 18th Conference on Air-Sea Interaction, 2012, Boston. Proceedings of the 20th Symposium on Boundary Layer Turbulence / 18th Conference on Air-Sea Interaction, 2012. 
\item[] ACEVEDO, O. ; PEZZI, L. P. . Estimating the energy released from a mesoscale oceanic eddy to the atmosphere from in situ vertical profiles of scalars. In: 20th Symposium on Boundary Layer Turbulence / 18th Conference on Air-Sea Interaction, 2012, Boston. Proceedings of the 20th Symposium on Boundary Layer Turbulence / 18th Conference on Air-Sea Interaction, 2012. 
\item[] SCHULTZ, C. ; PEZZI, L. P. . Estimation of Air-Sea Carbon Dioxide Flux at the Southwestern Atlantic Ocean using a circulation and biogeochemical model. In: 20th Symposium on Boundary Layers and Turbulence/18th Conference on Air-Sea Interaction, 2012, Boston. Proceedings of 20th Symposium on Boundary Layers and Turbulence/18th Conference on Air-Sea Interaction, 2012. 
\item[] PARISE, C. K. ; PEZZI, L. P. . Sensitivity of a fully coupled model to an extreme increase in the thickness and concentration of the Antartic Sea Ice. In: XXXII SCAR and Open Science Conference \& COMNAP XXIV AGM, 2012, Portland. Proceedings of XXXII SCAR and Open Science Conference \& COMNAP XXIV AGM, 2012. 
\item[] SCHULTZ, C. ; PEZZI, L. P. ; FERREIRA, W. . Biogeochemistry implementation in a regional ocean circulation model at the Brazil‐Malvinas confluence region: analysis for 1987‐1996 period. In: 10th ICSHMO International Conference on Southern Hemisphere Meteorology and Oceanography, 2012, Noumea. Proceedings on 10th ICSHMO International Conference on Southern Hemisphere Meteorology and Oceanography, 2012. 
\item[] SCHULTZ, C. ; PEZZI, L. P. . Atlantic Carbon Experiment (Acex): results from the first cruise. In: 10th ICSHMO International Conference on Southern Hemisphere Meteorology and Oceanography, 2012, Noumea. Proceedings of 10th ICSHMO International Conference on Southern Hemisphere Meteorology and Oceanography, 2012. 
\item[] PARISE, C. K. ; PEZZI, L. P. . Memória do sistema climático acoplado a uma condição extrema de concentração de gelo marinho na Antártica. In: I Encontro de Pesquisadores do INCT da Criosfera, 2012, Nova Petropolis. Anais doI Encontro de Pesquisadores do INCT da Criosfera. Porto Alegre: UFRGS, 2012.
\item[] DIAS, D. F. ; GHERARDI, Douglas F. M. ; PEZZI, L. P. . Modelling biophysical interactions: preliminary results on dynamics of the Southeast Brazil Bight using ROMS. In: 2012 ROMS/TOMS User Workshop, 2012, Rio de Janeiro. 2012 ROMS/TOMS User Workshop., 2012. 
\item[] SOARES, H. C. ; GHERARDI, Douglas F. M. ; PEZZI, L. P. . Assessment of climate variability impacts on the Brazilian Large Marine Ecosystems using statistical analysis and regional ocean modeling. In: 2012 ROMS/TOMS User Workshop, 2012, Rio de janeiro. 2012 ROMS/TOMS User Workshop, 2012. 
\item[] RUSSO, L. ; PEZZI, L. P. ; SAPUCCI, Luiz F . Impactos da Assimilação dos perfis de radiossondagens atmosféricas na região da Confluência Brasil-Malvinas. In: Simpósio Brasileiro sobre Pesquisa Antártica, 2008, São Paulo. Anais do XVI SPA. São Paulo: Instituto de Geociências, 2008. v. 1. p. 51-51. 
\item[] SILVEIRA, I. P. ; PEZZI, L. P. ; MATA, Mauricio M . Estudo das anomalias atmosféricas e oceânicas na região da Confluência Brasil-Malvinas. In: Simpósio Brasileiro sobre Pesquisa Antártica, 2008, São Paulo. Anais do XVI SPA. São Paulo: Instituto de Geociências - USP, 2008. v. 1. p. 49-49. 
\item[] SOPPA, M. ; PEZZI, L. P. . Variabilidade da temperatura da superfície do mar no Atlântico Sudoeste e sua relação com o fenômeno El Niño Oscilação Sul. In: Simpósio Brasileiro sobre Pesquisa Antártica, 2008, São Paulo. Anais do XVI SPA. São Paulo: Instituto de Geociências, 2008. v. 1. p. 55-55. 
\item[] RUSSO, L. ; PEZZI, L. P. . Estimativas da estabilidade atmosférica na região da Confluência Brasil-Malvinas entre 2004-2007. In: Simpósio Brasileiro sobre Pesquisa Antártica, 2008, São Paulo. Anais do XVI SPA. São Paulo: Instituto de Geociências - USP, 2008. v. 1. p. 134-134. 
\item[] PEZZI, L. P.; ACEVEDO, O. ; GARCIA, Carlos A E ; MATA, Mauricio M ; WAINER, Ilana E. K. ; CAMARGO, Ricardo de . Resultado de quatro campanhas do projeto interação Oceano-Atmosfera na Região da Confluência Brasil-MAlvinas (INTERCONF). In: XVI Simpósio brasileiro sobre Pesquisa Antártica, 2008, São Paulo. Anais do XVI SPA. São Paulo: Instituto de Geociências - USP, 2008. v. 1. p. 52-52. 
\item[] PEZZI, L. P. ; CAMARGO, Ricardo de ; MATA, Mauricio M ; GARCIA, Carlos A E . Programa INTERCONF: Observações in situ e variabilidade do sistema Oceano-Atmosfera na região da Confluência Brasil-Malvinas. In: Simpósio Brasileiro sobre Pesquisa Antártica, 2008, São Paulo. XVI SPA. São Paulo: Instituto de Geociências, 2008. v. 1. p. 30-30. 
\item[] PEZZI, L. P.; ARAVÉQUIA, José Antonio ; SAPUCCI, Luiz F ; HERDIES, Dirceu L ; TOMITA, Simone S ; CAMARGO, Ricardo de ; SOUZA, Ronald Buss de . Perspectivas para a assimilação de dados de radiossondagens na Região da Confluência Brasil-Malvinas por modelos atmosféricos regionais e global.. In: XIV Simpósio Brasileiro Sobre Pesquisa Antártica, 2006, São Paulo. Programa e Resumos do XIV Simpósio Brasileiro Sobre Pesquisa Antártica, 2006. p. 47-47. 
\item[] PEZZI, L. P.; SOUZA, Ronald Buss de . Ocean-Atmosphere interactions at the Brazil-Malvinas Confluence Region. In: XIV Simpósio Brasileiro Sobre Pesquisa Antártica, 2006, São Paulo. Programa e Resumos do XIV Simpósio Brasileiro Sobre Pesquisa Antártica, 2006. p. 48-49. 
\end{itemize}

\end{cvletter}
\end{document}
